\documentclass{article}
\usepackage{fullpage}
\usepackage{hyperref}
\begin{document}
    \title{150 Smoots Getting Started on Linux}
    \author{Andrew Chen}
    \date{\today}
    \maketitle
	\section{Repository Access}
		\begin{itemize} 
			\item \textbf{Install} Git packages with \emph{sudo apt-get install git-core git-gui git-doc}. Set up your basic git settings \href{http://help.github.com/git-email-settings/}{like so}.
			\item \textbf{Sign up} for a GitHub account (\href{https://github.com/signup/free}{here}) then ask Russell to add your account as a collaborator to the rcoh/SmootLight code repository.
			\item \textbf{Connect} to the SmootLight GitHub code repository. Generate SSH keys and upload the public one to GitHub by following the instructions \href{http://help.github.com/linux-key-setup/}{here}.
			\item \textbf{Clone and branch} the code repository; use \href{http://help.github.com/git-cheat-sheets/}{this cheatsheet} for help as necessary.
		\end{itemize}
	\section{Testing on Pygame}
		\begin{itemize} 
			\item \textbf{Install} Pygame with \emph{sudo apt-get install python-pygame}.
			\item \textbf{Setup} logging by running \emph{./setup.sh}.
			\item \textbf{Run} your LightInstallation code with \emph{python LightInstallation.py}.
		\end{itemize}
	\section{Repository Check In}
		\begin{itemize}
			\item \textbf{Stage} files to be committed with \emph{./ga}.
			\item \textbf{Commit} your changes locally with \emph{git commit}.
			\item \textbf{Push} your changes to your branch with \emph{git push origin yourbranchname}.
			\item \textbf{Request a pull} to the source repository by following instructions \href{http://help.github.com/pull-requests/}{here}.
			\item \textbf{Learn more} about working with remote repositories \href{http://help.github.com/remotes/}{here}.
		\end{itemize}
	\section{Latex Documentation}
        \begin{itemize} 
			\item \textbf{Install} Latex packages to be consistent with current documentation style. I had to run \emph{sudo apt-get install texlive-fonts-recommended texlive-latex-extra} to successfully convert Russell's .tex to pdf.
			\item \textbf{Edit} the .tex files with any text editor. I use vim for quick edits and gedit with the Latex plugin for autocompletion and quick previews. Get the Latex plugin with \emph{sudo apt-get install gedit-latex-plugin}.
            \item \textbf{Publish to PDF} with \emph{pdflatex yourfile.tex}.
			\item \textbf{View PDF} with \emph{evince yourfile.pdf}.
		\end{itemize}
\end{document}
